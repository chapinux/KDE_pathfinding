\documentclass[10pt,a4paper,french]{article}
\usepackage{graphicx} % Required for including pictures
\usepackage[T1]{fontenc}
\usepackage[utf8]{inputenc}
\usepackage[french]{babel}
\usepackage[dvipsnames]{xcolor}
\usepackage{lmodern}
\renewcommand*\familydefault{\sfdefault} 
\definecolor{IGN_green}{rgb}{0.549,0.741,0.227}
\definecolor{IGN_green_dark}{rgb}{0.273,0.368,0.113}
\newcommand{\LASTIG}{\href{https://www.umr-lastig.fr/}{\bf LASTIG}}
\newcommand{\ENSG}{\href{https://www.ensg.eu/}{\bf ENSG}}
\newcommand{\IGN}{\href{https://ign.fr/}{\bf IGN}}
\newcommand{\UnivGustaveEiffel}{\href{https://www.univ-gustave-eiffel.fr/}{\bf Université Gustave Eiffel}}
\newcommand{\ACTE}{\href{https://www.umr-lastig.fr/acte/}{ACTE}}
\newcommand{\GEOVIS}{\href{https://www.umr-lastig.fr/geovis/}{GEOVIS}}
\newcommand{\MEIG}{\href{https://www.umr-lastig.fr/meig/}{MEIG}}
\newcommand{\STRUDEL}{\href{https://www.umr-lastig.fr/strudel/}{STRUDEL}}
\usepackage{geometry}
\geometry{left=20mm,right=20mm,bindingoffset=0mm,top=20mm,bottom=20mm}
\usepackage{titlesec}
\titleformat*{\section}{\center\Large\bfseries\color{IGN_green_dark}}
\usepackage{multirow}
\usepackage{hyperref}
\hypersetup{
    unicode=false,             % non-Latin characters in Acrobat’s bookmarks
    pdftoolbar=true,           % show Acrobat’s toolbar?
    pdfmenubar=true,           % show Acrobat’s menu?
    pdffitwindow=false,        % window fit to page when opened
    pdfstartview={FitH},       % fits the width of the page to the window
    pdftitle={Sujet Projet AS IGAST: KDE \& pathfinding }, % title
    pdfnewwindow=true,         % links in new PDF window
    colorlinks=true,           % false: boxed links; true: colored links
    linkcolor=red,             % color of internal links (change box color with linkbordercolor)
    citecolor=green,           % color of links to bibliography
    filecolor=magenta,         % color of file links
    urlcolor=IGN_green_dark         % color of external links
}
\begin{document}
\includegraphics[height=0.8cm,trim={20 20 20 20},clip]{logo_lastig.png}
\hfill
\includegraphics[height=0.8cm,trim={10 10 10 10},clip]{logo_IGN-ENSG.png}
\hfill
\includegraphics[height=0.8cm]{logo_UGE.png}

\vspace{0.5em}

\begin{center}
\uppercase{\Large Estimation de la densité 2D par noyau dans l'espace urbain «marchable» }
\end{center}

\hrule
\begin{center}
\bf \uppercase{Sujet de Projet d'Analyse Spatiale, M2 IGAST}
\end{center}
\hrule
\begin{center}
\begin{tabular}{l|l}
     {\bf Intitulé du poste} & Ingénieur\textperiodcentered e de recherche (ou d’études pour la recherche)\\
     &\\
     {\bf Site / Lieu de travail} & LASTIG - 73 avenue de Paris - 94160 - Saint-Mandé\\
    % {\bf Type de contrat} & XXX\\
    % {\bf Salaire } & YYY\\
%     {\bf Date de recrutement} & 1 septembre 2022\\
     {\bf Niveau d’étude requis} & Diplôme d'ingénieur ou doctorat en informatique\\
\end{tabular}
\end{center}

% \section*{Contexte}

% L’\ENSG, chargée de la politique de recherche et d’enseignement de l'\IGN, regroupe les services de l’école, proprement dite, sur le campus de l’\UnivGustaveEiffel~et les équipes de recherche en sciences de l’information géographique, majoritairement implantées à Saint-Mandé et regroupées sous l’autorité fonctionnelle du \LASTIG.
% Le centre de compétences \emph{Technologies des Systèmes d’Information} (CC TSI) regroupe une dizaine d’agents et comprend un pôle d’enseignement qui regroupe les personnels enseignants et enseignants-chercheurs et dont le responsable est localisé à Champs-sur-Marne.
% Le CC TSI participe aux programmes de formation, de recherche, aux missions d'expertise, à la valorisation de la recherche (avec le Service de l’innovation) et à la gestion des équipements ou plateformes qui soutiennent les activités de recherche ou d’enseignement.

% \smallskip
% L’Unité Mixte de Recherche (UMR) \LASTIG, sous la tutelle de l’\IGN~et de l’\UnivGustaveEiffel, mène des recherches variées en sciences de l’information géographique pour la ville durable et les territoires numériques. Il comporte plus de 60 membres, permanents et contractuels. Le \LASTIG~est confronté à des problématiques de recherche fondamentale et opérationnelle sur les sujets suivants :
% \begin{itemize}
% \item l’acquisition et le traitement de données massives et multimodales (équipe \ACTE) ;
% \item la géovisualisation, l’interaction et l’immersion (équipe \GEOVIS) ;
% \item la médiation et l'enrichissement de données géographiques (équipe \MEIG) ;
% \item l’analyse de la dynamique spatio-temporelle des territoires (équipe \STRUDEL).
% \end{itemize}

\section*{Missions}

Le \LASTIG~a de nombreux besoins en terme d'infrastructures de recherche et souhaite se doter d'un·e ingénieur·e spécialisé·e dans la mise en place et la maintenance d'infrastructures partagées de recherche (service et données).
Cet·te Ingénieur·e pourra, par ailleurs, appuyer la politique de gestion des données de la recherche du laboratoire.
\textcolor{cyan}{\bf{Une demande de création d'un poste similaire plus spécifiquement orienté sur le calcul a été initiée, avec pour objectif de couvrir l'ensemble des besoins du \LASTIG~grâce à ce binôme. }}

\medskip
Les missions principales de la personne recrutée seront de~:

\begin{itemize}
    \item Faire l'\textbf{inventaire des infrastructures de service et de données} utilisées et/ou développées par des chercheur-e-s du \LASTIG.
    \item Proposer un \textbf{état de l'art} de telles infrastructures (nationales et internationales) pertinentes pour le laboratoire.
    \item Contribuer à la politique de \textbf{mise en \oe uvre} des infrastructures partagées du LASTIG \textcolor{cyan}{\bf{notamment avec les constituants de l'UGE}}.
    \item Faire l'\textbf{inventaire des données de recherche ouvertes} du laboratoire.
    \item Aider dans la mise en place de \textbf{plan de gestion des données}, \textcolor{cyan}{\bf{notamment dans le cadres de projets de recherche portés par le laboratoire}}.
    \item Participer à la \textbf{capitalisation sur les travaux et le code produits} lors de projets, thèses, stages après la fin de ceux-ci ou lors de leur dernière phase, et en lien avec les missions du poste.
    \item Participer à la \textbf{valorisation des activités de recherche} (y compris publications).
    \item Développer des \textbf{prototypes} et participer aux activités d’\textbf{innovation} de l’IGN utilisant les recherches du \LASTIG.
    \item Contribuer à la \textbf{communauté d’experts Infrastructure} (notamment au centre de compétences en informatique).
    \item Participer aux enseignements à l’ENSG.
\end{itemize}

% \medskip
% En particulier, deux projets transverses au \LASTIG~...~:

% \begin{enumerate}
%     \item Le premier projet a comme objectif ...
%     \item Le deuxième projet a comme objectif ... 
% \end{enumerate}

%L’accueil et l’encadrement fonctionnel se feront au LASTIG, à Saint-Mandé.

%\newpage
% \section*{Profil recherché}
% \subsection*{Compétences techniques}
% \begin{itemize}
%     \item Maîtriser des \emph{environnements et langages de programmation} (Python, Java, JavaScript, C++, etc.)
%     \item Avoir des notions en \emph{architecture de données massives} (Hadoop, etc.)
%     \item Maîtriser les \emph{outils DevOps} (intégration continue, etc.) et de \emph{développement Agile}
%     \item Maîtriser des \emph{outils libres en géomatique} (PostGreSQL/ PostGIS, QGIS, GDAL, etc.)
%     \item Savoir expertiser et assurer la \emph{qualité scientifique de résultats} 
%     \item Pratiquer un \emph{anglais technique et/ou scientifique}
%     \item Savoir \emph{rédiger des articles scientifiques}
% \end{itemize}

% \subsection*{Compétences organisationnelles}
% \begin{itemize}
%     \item Savoir \emph{transmettre} un savoir, une technique, une compétence
%     \item Savoir \emph{concevoir et mettre en \oe uvre} des solutions nouvelles et efficaces 
%     \item Savoir \emph{publier et diffuser} de l’information dans un système informatisé 
%     \item Savoir \emph{mener une veille} sur son domaine d’activité 
%     \item Avoir le \emph{sens de la responsabilité, de l’autonomie, de l’initiative}
% \end{itemize}

% \subsection*{Compétences relationnelles}
% \begin{itemize}
%     \item Savoir \emph{collaborer et travailler en équipe}
%     \item Savoir \emph{encourager la synergie et partager ses connaissances et ses expériences}
% \end{itemize}

% %\subsection*{Expérience professionnelle souhaitée ?}

% \section*{Conditions particulières d’exercice du poste}
% Possibilité de déplacements en conférences scientifiques pour la diffusion de recherches, possibilité de travail multi-sites à Champs-sur-Marne.
%\begin{itemize}
%    \item \makebox[4.5cm]{\bf Lieu de travail\hfill} IGN, LaSTIG – 73 avenue de Paris - 94160 - Saint-Mandé, proximité Métro (Saint-Mandé - ligne 1) et RER (Vincennes - ligne A)
%    \item \makebox[4.5cm]{\bf Type de contrat\hfill} ?
%    \item \makebox[4.5cm]{\bf Date de prise de poste\hfill} ?
%    \item \makebox[4.5cm]{\bf Rémunération\hfill} ? selon diplôme et expérience
%\end{itemize}

% \section*{Contacts}
% Merci d’adresser votre candidature (lettre de motivation + CV en un seul PDF) en précisant impérativement la référence~\textbf{LASTIG\_LIDAR\_HD} à l'ensemble des adresses suivantes~:
% \begin{center}
% \begin{tabular}{ll}
%     {\bf Ana-Maria Raimond}&\multirow{2}{*}{ana-maria(point)raimond(at)ign(point)fr}\\
%     {\small (co-directrice du \LASTIG, équipe \MEIG)}&\\
%     {\bf Julien Perret}&\multirow{2}{*}{julien(point)perret(at)ign(point)fr}\\
%     {\small (\LASTIG, responsable de l'équipe \STRUDEL)}&\\
%     {\bf Laurent Breton}&\multirow{2}{*}{laurent(point)breton(at)ensg(point)eu}\\
%     {\small (\ENSG, responsable du CC TSI)}&
% \end{tabular}
% \end{center}
%\section*{Liens}
%\href{https://www.umr-lastig.fr/}{Site web du laboratoire LASTIG}
\end{document}
